\documentclass[twocolumn,11pt]{jsarticle}

%\usepackage{kanjifonts}
%\usepackage{twocolumn}
\usepackage{amsmath}
\usepackage{ascmac}
\usepackage{geometry}
\usepackage{fancyhdr}
\geometry{body={167mm,230mm},columnsep=8mm}

%% \renewcommand{\labelitemi}{一.}
%% \renewcommand{\labelitemi}{□ }
%% \renewcommand{\labelitemii}{□ }
\usepackage{latexsym}
\renewcommand{\labelitemi}{$\Box$}
\renewcommand{\labelitemii}{$\Box$}
\renewcommand{\labelitemiii}{- }


\begin{document}

\twocolumn[
\begin{center}
  \textbf{\LARGE 新春恒例TODO}
  \vspace{8mm}
\end{center}
]

\setcounter{page}{1}


\section*{配布物}
\begin{itemize}
\item 研究室のくらし
\item ネットワークの図
\end{itemize}

\section{新春の作業}

\subsection{サーバ設定}
\begin{itemize}
\item 新メンバのアカウント確認
\item ML更新(bcl-lsc, gradbcl-lsc, 裏ページ参照)
%\item 古いユーザのメール着信拒否の設定
\item 古いユーザのデータ移行,アカウント停止
\item zeusの古いアカウント整理
\end{itemize}

\subsection{情報更新}
\begin{itemize}
\item ホームページのメンバー表更新
\item 名簿(members-info)更新
  \begin{itemize}
  \item 新メンバ情報入力
  \item 住所変更も確認
  \end{itemize}
\item 計算機一覧(裏ページ)更新
\item 係決め(裏ページ参照)
\end{itemize}

\subsection{新メンバへのガイダンス}

\begin{itemize}
\item googleカレンダーアカウント作り
\item slackアカウント作り
\item 研究室での生活について説明
  \begin{itemize}
  \item そうじの仕方\&掃除
    \begin{itemize}
    \item 掃除機
    \item ふき掃除
    \item ごみ捨て(分別方法、すてる場所)
    \end{itemize}
  \item 水回りの管理(冷蔵庫、お茶、etc)
  \item 生活費
  \item 喫煙場所
  \item etc?
  \end{itemize}
\item 備品(製本機,パンチ,ホッチキス等の場所...)の紹介
\item セミナー室(207,405)の紹介
\item コピーの方法
\end{itemize}


\subsection{パソコン講習・整備}
\begin{itemize}
%\item Windows機(ノートPC含む)のメンテナンス
\item 共用パソコンのメンテナンス
  \begin{itemize}
  \item Windows Upgrade
  \item ウィルス対応ソフトの更新
  \item Macの個人フォルダの整理
  \end{itemize}

\item パソコンの整備
  \begin{itemize}
  \item マシンのそうじ
  \item ケーブル作り
  \item OSのインストール(インストールマニュアル参照)
  \end{itemize}
\item 計算機環境講習会(Appendix\ref{sec:comp}参照)
\end{itemize}

\section{模様替え}
\begin{itemize}
\item 模様替えの計画(机・本箱その他もろもろの配置決定)
\item 席決め
\item 本箱整理(共有のものを整理し、新メンバーのスペース確保)
\item 新4年生本箱決め(テプラ)
\item 新4年生マシン決め
\item 下駄箱整理
  \begin{itemize}
  \item 古いのを処分
  \item 新メンバーの場所決定(テプラ)
  \end{itemize}
\item 棚や箱にあるものを整理
\end{itemize}

\section{新卒論生の新春行事}
\begin{itemize}
\item Vine LinuxをノートPCにインストール
\item 宿題チェック
\item 計測実習
\item 計測実習の発表準備
\end{itemize}

\section{各種スケジュール決定}

\begin{itemize}
\item そうじの日(美化係)
\item 実習発表会の日程決定
\item セミナーの日程
\item 本読みの日程
\end{itemize}

\appendix

% \section{係一覧\label{sec:kakari}}
% \begin{tabular}[t]{|ll|}\hline
% 取締役 &   \\
% 美化係 &   掃除責任者。掃除当番を決めて清掃状況の監督をする。\\
% イベント係 & イベント、コンパ、レクリエーション等の企画\\
% 掲示・広報係 &   本箱・下駄箱のテプラ作成。calendar(研究室、セミナー室)の情報登録・管理。\\
% &名簿、ML、研究室ホームページの管理。その他各種掲示・配布物作成等。 \\
% コンピュータ係 & バックアップその他計算機管理, ホームページ作り等\\\hline
% \end{tabular}


\section{研究室の計算機環境について\label{sec:comp}}
\begin{itemize}
\item 研究室内のネットワークについて(別紙)
  \begin{itemize}
  \item webサーバ
  \item メールサーバ
  \item NFSサーバ
  \item NISサーバ
  \item 計算サーバ
  \item Matlabの利用方法
  \end{itemize}
% \item quotaについて(「UNIX/Linuxの基本操作」参照)
%   \begin{itemize}
%   \item quotaの目的
%     \begin{itemize}
%     \item 異常ファイルや不正アクセスによる被害を食い止める。
%     \item 不要なファイルによるディスクの食いつぶしを防ぐ
%     \end{itemize}
%   \item quotaにひっかかるとどういう症状になるか?
%   \item 利用ディスク容量を小さくするには(不要ファイルの圧縮・削除)
%   \item quota拡大は計算機係に
%   \end{itemize}
\item バックアップのしくみとスケジュール
\item シャットダウンの仕方について
\item 無線LANについて
\item 研究室のホームページについて
  \begin{itemize}
  \item \verb|http://bcl.sci.yamaguchi-u.ac.jp/|
  \item 各種ドキュメントも公開しています
%   \item 作りなおし予定(ホームページ係)
  \item 修正に御協力ください
  \end{itemize}
\item 非常時の対応(「UNIX/Linuxの基本操作」の「トラブル!」参照)
%\item 個人のWebページのつくりかた
%  \begin{itemize}
%  \item \verb|~/public_html|以下のファイルをいじって \verb|make install|。
%    これでwebサーバ(zeus)上にファイルがコピーされ、外部に公開されま
%    す。
%  \item 各自のページをつくってみよう。(用意されているindex.htmlを修正
%  してmake installする)
%  \end{itemize}

\item 研究室の裏ページについて
  \begin{itemize}
  \item \verb|http://bcl.sci.yamaguchi-u.ac.jp/lab/|
  \item 表ページの下にあるリンクからたどれます
  \item 要パスワード
  \item だれでもいじれます。いろいろな情報追加に協力おねがいします。
  \item 裏ページのコンテンツ紹介
  \item ギャラリーも活用してください(裏と同じパスワードで多分登
    録可)
  \item ちょっといじってみよう!(仮の名前のページを各自ちょっと作ってみ
  る)
  \end{itemize}
\item その他
\end{itemize}

\end{document}
