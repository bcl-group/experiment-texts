\documentclass{jarticle}
\usepackage[hypertex]{hyperref}
\usepackage{amsmath}
\usepackage{ascmac}
\usepackage[height=26cm,width=18cm]{geometry}
\usepackage{fancyhdr}

\begin{document}
\title{VENUS, ZEUSバックアップ・リストア}
\date{\today}
\author{谷 合 由 章}
\topmargin -15mm        \textheight 250mm       \textwidth 180mm
\oddsidemargin -10mm    \evensidemargin -10mm
\maketitle

\begin{quote}
\begin{screen}
\tableofcontents
\end{screen}
\end{quote}

\section{TODO}

\begin{itemize}
\item システムバックアップをリストアする方法.
\item partimage の使用上の注意(リストアする場合リストア先の
      パーティションはバックアップしたときのパーティションの
      サイズ以上でないとリストアできない)..
\item パーティションにデータをリストアしたとき, リストアした
      パーティションのサイズは, バックアップしたときのパーテ
      ィションサイズになってしまうこと(注意)
\item MBRのリストア方法
\end{itemize}

\section{最初に}
現在研究室のコンピュータ設備の中心的役割を行っているマシンがあ
る. venus(ファイルサーバ) と zeus(メールとウェブサーバ) だ. こ
れらのマシンが故障しているとき, さまざまな業務が行えない.

マシンの故障はさまざまだ.

\begin{itemize}
\item ハードディスクの破損
\item ウィルスによるシステム・データの破壊
\item MBR(Master Boot Record)の破損
\end{itemize}
各状況により, バックアップの操作は変わる(必要な復旧作業のみで
時間を節約できることがある). 例えば, MBRが壊れたとわかれば,
MBRだけを復旧すれば良く, 全部のシステムとデータを入れ直す必要
はない. また, ハードディスクが破損したら, 新しいハードディスク
を用意し, データを入れなおす必要がある.

この文書では, システムファイルを含むハードディスクが使用できなく
なった場合の復旧方法, 復旧のためのバックアップの仕方を説明する.
被害が少ない場合, 復旧方法の途中まで読み飛ばして欲しい.

\section{かならず必要なもの}

あらかじめ必要なものは, 作業に取りかかる前にかならず準備して行おう.

\begin{enumerate}
\item 研究室の住人に作業日の連絡をしよう(ファイルを変更中, プログラムを
      実行している人がいるかもしれない).
\item /home/venus と /home/zeus のバックアップを行う(後程解説
      する).
\item バックアップしたデータを保存するためのDVDを手元に用意しよう.
\item System Rescue CD-ROMを手元に用意しよう.
\item 十分な時間を用意して作業しよう. これは慌てない状況を作り, 思い
      もよらない状況でもしっかり対処するためである. 中途半端な作業を
      時間をおいてくり返し, 数日に渡らないようにしよう.

\item 最低一回は venus, zeus以外のマシンでこの紙に書かれている
      ことを参考にバックアップを行い, リストアをしてみよう. と
      りあえずしてみよう, と思えないほど venus, zeusは大切である.
      みんなが研究に使ったの何週間, 何ヶ月の時間がなかったことに
      なるうえ, メールによる連絡もできなくなる.
\end{enumerate}
以上の一つでもかけるようなら, 作業をしないこと.

\section{パーティション情報 \label{sec:prttn}}

venus と zeus の情報は, 復旧の際に必要となるので以下に情報を載せておく.

\subsection{venus編}

\begin{verbatim}
[venusのマウント情報]

ファイルシステム    1k-ブロック   使用中      空き 使用% マウント場所
/dev/hda5              5036284   2242520   2537932  47% /
/dev/hda1                38856     16775     20075  46% /boot
/dev/hda2             10080520   3201700   6366752  34% /SB
/dev/hda3             10080488     44740   9523680   1% /free2
/dev/sda6             10080488     33408   9535012   1% /home
/dev/hda7             12839868     32812  12154820   1% /free3
none                    192420         0    192420   0% /dev/shm
/dev/sda1             50403000   2527852  45314792   6% /home/venus
/dev/sda2             50403028   3608920  44233752   8% /home/venus-bkup
/dev/sda3             10080520    737528   8830924   8% /home/zeus
/dev/sda5             10080488    407300   9161120   5% /home/zeus-bkup
/dev/sda7             20161172   2270016  16867016  12% /home/public
/dev/sda8              5044156   1695748   3092176  36% /vinecd-make
/dev/sda9            139271376     32828 132163928   1% /raid-free-space
\end{verbatim}

\begin{verbatim}
[venusのパーティション情報]

[ディスク /dev/hda]: ヘッド 255, セクタ 63, シリンダ 4865
ユニット = シリンダ数 of 16065 * 512 バイト

 デバイス ブート   始点      終点  ブロック   ID  システム
/dev/hda1   *         1         5     40131   83  Linux
/dev/hda2             6      1280  10241437+  83  Linux
/dev/hda3          1281      2555  10241437+  83  Linux
/dev/hda4          2556      4865  18555075    f  Win95 拡張領域 (LBA)
/dev/hda5          2556      3192   5116671   83  Linux
/dev/hda6          3193      3241    393561   82  Linux スワップ
/dev/hda7          3242      4865  13044748+  83  Linux

[ディスク /dev/sda]: ヘッド 255, セクタ 63, シリンダ 37378
ユニット = シリンダ数 of 16065 * 512 バイト

 デバイス ブート   始点      終点  ブロック   ID  システム
/dev/sda1             1      6375  51207156   83  Linux
/dev/sda2          6376     12750  51207187+  83  Linux
/dev/sda3         12751     14025  10241437+  83  Linux
/dev/sda4         14026     37378 187582972+   5  拡張領域
/dev/sda5         14026     15300  10241406   83  Linux
/dev/sda6         15301     16575  10241406   83  Linux
/dev/sda7         16576     19125  20482843+  83  Linux
/dev/sda8         19126     19763   5124703+  83  Linux
/dev/sda9         19764     37378 141492456   83  Linux
\end{verbatim}

\subsection{zeus編}

\begin{verbatim}
[zeusのマウント情報]

ファイルシステム    サイズ 使用中 空き 使用% マウント場所
/dev/hda2             2.9G  2.3G  468M  84% /
/dev/hda1              23M  7.3M   14M  34% /boot
/dev/hda5              99M  4.1M   89M   5% /free1
/dev/hda6              33G  794M   30G   3% /home2
none                  251M     0  250M   0% /dev/shm
venus:/home/zeus      9.6G  721M  8.4G   8% /home
\end{verbatim}

\begin{verbatim}
[zeusのパーティション情報]

[ディスク /dev/hda]: 40.9GB, ヘッド 255, セクタ 63, シリンダ 4982
ユニット = シリンダ数 of 16065 * 512 バイト = 8225280 バイト

 デバイス ブート   始点      終点  ブロック   ID  システム
 /dev/hda1   *         1        13    104391   83  Linux
 /dev/hda2            14       500   3911827+  83  Linux
 /dev/hda3           501       623    987997+  82  Linux スワップ
 /dev/hda4           624      4982  35013667+   5  拡張領域
 /dev/hda5           624       636    104391   83  Linux
 /dev/hda6           637      4982  34909213+  83  Linux
\end{verbatim}

\section{バックアップ}

バックアップは人のいないときに行おう. 深夜, 土日などの休日が良い
だろう. しかし, zeus は ウェブサーバとして外部にサービスを提供し
ていることを考えると, 休日でも昼間は避けるのがよいだろう.

\subsection{venus のシステムバックアップ}

\begin{enumerate}
\item venusのサービスを利用しているマシン(zeus等) をシ
      ャットダウンする.
\item venus をシャットダウンする.
\item System Rescue CD-ROMを入れる(DVDドライブの方に).
\item venus マシンの電源を入れる.
\item System Rescue CD-ROMのシステムが立ち上がったら, とりあえず
      Enter を押す.
\item 次にキーボードのコードの選択を求められる. 日本語 22(ja)を
      リストから選ぼう.
\item プロンプトがでてくるので作業開始. /dev/hda2はバックアップイ
      メージを格納するためにマウントする(/, /boot, /home, swap 以
      外のパーティション).

      \begin{verbatim}
      # cd /mnt
      # mkdir free
      # mount -t ext3 /dev/hda2 free
      # partimage
      \end{verbatim}

\item $partimage$ を起動したら, 画面が変わる. まず /boot の
      バックアップから始めよう.

      \begin{enumerate}
      \item 「Partition to save/restore」では「ide/host0/bus0
             /target0/lun0/part1」を選ぼう(/boot領域であるか
             確かめよう).
      \item 「Image file to create/use」では「/mnt/free/venus
             -boot.img」と記入する.
      \item 「Action to be done:」では「Save partition into a
             new image file.」を選ぼう.
      \item [F5]を押して, 次に進む.
      \item 「Compression level」では「Gzip」の欄を選択しよう.
      \item [F5]で次に進む.
      \item 「Partition description」では何も入力せず, [OK]
            を押す.
      \item バックアップを行う内容が表示されるので, 確認しよう.
            間違いがないなら, [Enter]を押す(これでバックアップが
            開始される).
      \item 完了するまで待機(10分もかからず終わるだろう).
      \end{enumerate}

\item 次に /(ルートディレクトリ) のバックアップを行う.
      $partimage$ を実行しよう.

      \begin{enumerate}
      \item 「Partition to save/restore」では「ide/host0/bus0
             /target0/lun0/part5」を選ぼう(/領域であるか確かめ
             よう).
      \item 「Image file to create/use」では「/mnt/free/venus
             -root.img」と記入する.
      \item 「Action to be done:」では「Save partition into a
             new image file.」を選ぼう.
      \item [F5]を押して, 次に進む.
      \item 「Compression level」では「Gzip」の欄を選択しよう.
      \item [F5]で次に進む.
      \item 「Partition description」では何も入力せず, [OK]
            を押す.
      \item バックアップを行う内容が表示されるので, 確認しよう.
            間違いがないなら, [Enter]を押す(これでバックアップが
            開始される).
      \item 完了するまで待機(30分ぐらいかかる).
      \end{enumerate}

\item 最後に /home のバックアップを行う.
      $partimage$ を実行しよう.

      \begin{enumerate}
      \item 「Partition to save/restore」では「ide/host0/bus0
             /target0/lun0/part6」を選ぼう(/home 領域であるか
             確かめよう).
      \item 「Image file to create/use」では「/mnt/free/venus
             -home.img」と記入する.
      \item 「Action to be done:」では「Save partition into a
             new image file.」を選ぼう.
      \item [F5]を押して, 次に進む.
      \item 「Compression level」では「Gzip」の欄を選択しよう.
      \item [F5]で次に進む.
      \item 「Partition description」では何も入力せず, [OK]
            を押す.
      \item バックアップを行う内容が表示されるので, 確認しよう.
            間違いがないなら, [Enter]を押す(これでバックアップが
            開始される).
      \item 完了するまで待機(30分ぐらいかかる).
      \end{enumerate}
\item マウントしているファイルシステムを unmount してから, シャ
      ットダウンを行おう.

      \begin{verbatim}
      # umount free
      # shutdown -h now
      \end{verbatim}

\item System Rescue CD-ROMを取り除いて, venusを起動させよう.
\end{enumerate}

\subsection{zeus のシステムバックアップ}

\begin{enumerate}
\item zeus をシャットダウンする.
\item System Rescue CD-ROMを入れる.
\item zeus マシンの電源を入れる.
\item System Rescue CD-ROMのシステムが立ち上がったら, とりあえず
      Enter を押す.
\item 次にキーボードのコードの選択を求められる. 日本語(ja)を
      リストから選ぼう.
\item プロンプトがでてくるので作業開始. /dev/hda2はバックアップイ
      メージを格納するためにマウントする(/, /boot 以外のパーティシ
      ョン).

      \begin{verbatim}
      # cd /mnt
      # mkdir home2
      # mount -t ext3 /dev/hda6 home2
      # partimage
      \end{verbatim}

\item $partimage$ を起動したら, 画面が変わる. まず /boot の
      バックアップから始めよう.

      \begin{enumerate}
      \item 「Partition to save/restore」では「ide/host0/bus0
             /target0/lun0/part1」を選ぼう(/boot領域であるか
             確かめよう).
      \item 「Image file to create/use」では「/mnt/home2/zeus
             -boot.img」と記入する.
      \item 「Action to be done:」では「Save partition into a
             new image file.」を選ぼう.
      \item [F5]を押して, 次に進む.
      \item 「Compression level」では「Gzip」の欄を選択しよう.
      \item [F5]で次に進む.
      \item 「Partition description」では何も入力せず, [OK]
            を押す.
      \item バックアップを行う内容が表示されるので, 確認しよう.
            間違いがないなら, [Enter]を押す(これでバックアップが
            開始される).
      \item 完了するまで待機(数分もかからず終わるだろう).
      \end{enumerate}

\item 次に /(ルートディレクトリ) のバックアップを行う.
      $partimage$ を実行しよう.

      \begin{enumerate}
      \item 「Partition to save/restore」では「ide/host0/bus0
             /target0/lun0/part3」を選ぼう(/領域であるか確かめ
             よう).
      \item 「Image file to create/use」では「/mnt/home2/zeus
             -root.img」と記入する.
      \item 「Action to be done:」では「Save partition into a
             new image file.」を選ぼう.
      \item [F5]を押して, 次に進む.
      \item 「Compression level」では「Gzip」の欄を選択しよう.
      \item [F5]で次に進む.
      \item 「Partition description」では何も入力せず, [OK]
            を押す.
      \item バックアップを行う内容が表示されるので, 確認しよう.
            間違いがないなら, [Enter]を押す(これでバックアップが
            開始される).
      \item 完了するまで待機(30分ぐらいかかる).
      \end{enumerate}

\item マウントしているファイルシステムを unmount してから, シャ
      ットダウンを行おう.

      \begin{verbatim}
      # umount home2
      # shutdown -h now
      \end{verbatim}

\item System Rescue CD-ROMを取り除いて, zeusを起動させよう.
\end{enumerate}

\subsection{venus の /home/venus をバックアップする \label{sub:vns-b}}

/home/venus ディレクトリの中には, ユーザディレクトリがあります. ユーザの
データをバックアップする作業の手順を以下に示します(シェルスクリプトを覗い
てどんな動作をしているか確認してください).

\begin{enumerate}
\item 十分に容量の残っているDVD$\pm$Rを venus のDVDドライブに
      入れる.
\item venus にログインする.
\item cd /home/venus-bkup/bkup
\item ./dvd-burn.sh を実行する.

      詳しくは, dvd-burn.sh の中身を見てください(使い方を書いています).

\item あとは待つだけ.
\end{enumerate}

\subsection{venus の /home/zeus をバックアップする \label{sub:zs-b}}

/home/zeus ディレクトリの中は, 研究室とそのユーザのホームページとして
公開しているデータがあります. 作業手順を下に示します(シェルスクリプト
を覗いてどんな動作をしているか確認してください).

\begin{enumerate}
\item 十分に容量の残っているDVD$\pm$Rを venus のDVDドライブに
      入れる.
\item venus にログインする.
\item cd /home/zeus-bkup/bkup
\item ./dvd-burn.sh を実行する.

      詳しくは, dvd-burn.sh の中身を見てください(使い方を書いています).

\item あとは待つだけ.
\end{enumerate}

\subsection{venus の /home/public をバックアップする \label{sub:pc-b}}

/home/public ディレクトリの中は, 研究に使うソフトウェアや, 研究を補助
するためのソフトウェアが多く含まれています(その他, 研究室の住人につい
ての情報, や研究室の情報がある). 作業手順を下に示します(シェルスクリプト
を覗いてどんな動作をしているか確認してください).

\begin{enumerate}
\item 十分に容量の残っているDVD$\pm$Rを venus のDVDドライブに
      入れる.
\item venus にログインする.
\item cd /home/venus/adm/bin/
\item ./dvd-burn-public.sh を実行する.

      詳しくは, dvd-burn-public.sh の中身を見てください(使い方を書いています).

\item あとは待つだけ.
\end{enumerate}

\subsection{最新のデータをバックアップするには}

/home/public だけは, 上記の方法で 最新のデータをバックアップできます.
ここでは, /home/venus と /home/zeus の最新のデータをバックアップする
ための方法を説明します. 以下も root での作業となります.

\begin{enumerate}
\item /home/venus/adm/bin/bclbkup を実行する.
\item /home/venus/adm/bin/bkup\_venus(もしくは bkup\_zeus) を実行する.

      このスクリプトを実行すると venus マシンに過大な負荷を与えるので,
      住人がいないときにすることをお薦めします.
\end{enumerate}
あとは, 上記のとおりにバックアップしてください.

\section{リストア手順}

すでにハードディスク上に適切なパーティションが作成されて
いるとして, 話しを進める(パーティションの作成とファイルシ
ステムのフォーマットは付録の第 \ref{sec:mkprttn-frmt}章を
参照のこと). リストア先のパーティションのデータは書きかえられ,
消えてしまいますので, リストア先を何回も確かめ 間違えのないように.

\begin{enumerate}
\item リストアのためのイメージファイルを用意しておく.

      磁気記憶装置に記憶させておくとよい.

\item System Rescue CD-ROMを入れる.
\item マシンの電源を入れる.
\item System Rescue CD-ROMのシステムが立ち上がったら, とりあえず
      Enter を押す.
\item 次にキーボードのコードの選択を求められる. 日本語(ja)を
      リストから選ぼう.
\item プロンプトがでてくるので作業開始. バックアップイメージを格納
      してあるパーティションをマウントする.

      \begin{verbatim}
      [例]

      # cd /mnt
      # mkdir home2
      # mount -t ext3 /dev/hda6 home2   # /dev/hda6の適切に置き換えよう.
      # partimage
      \end{verbatim}

\item $partimage$ を起動したら, 画面が変わる. まず /boot の
      リストアから始めよう.

      \begin{enumerate}
      \item 「Partition to save/restore」では「ide/host0/bus0
             /target0/lun0/part1」を選ぼう(/boot領域であるか
             確かめよう).
      \item 「Image file to create/use」では「/mnt/home2/zeus
             -boot.img.000」と記入する.
      \item 「Action to be done:」では「Restore partition from an
             image file.」を選ぼう.
      \item [F5]を押して, 次に進む.
      \item もう一度[F5]を押して, 次に進む.
      \item 「Partition description」では何も入力せず, [OK]
            を押す.
      \end{enumerate}

\item /boot のリストアと同様に必要なリストアを行う.
%\item 最後に MBR(マスタブートレコード) のリストアを行って,
%      リストアしたシステムを起動できるようにする.
%
%      \begin{enumerate}
%      \item  $partimage$を起動.
%      \item 「Action to be done:」では「Restore an MBR from the imagefile」
%             を選ぼう.
%      \item  [F5]を押して, 次に進む.
%      \item 「Disk with the to restore」では「/dev/ide/host0/bus0/target0/lun0/disc」
%             を選択する.
%      \item 「What to resore:」では「The Whole MBR」を選択する.
%      \end{enumerate}
%\item データのリストアが完了したら, リストアしたOS(operation system)
%      を起動できるようにする.
%
%      \begin{enumerate}
%      \item cd /mnt
%      \item mkdir /root
%      \item 磁気記憶装置(HDD)の root ディレクトリを /mnt/root にマウントする.
%
%           \begin{verbatim}
%           [例] 「/」ディレクトリが /dev/hda5 の場合
%
%           \item mount -t ext3 /dev/hda5 /mnt/root
%            \end{verbatim}
%
%      \item cd /mnt/root
%      \item 磁気記憶装置(HDD)の boot ディレクトリを /mnt/root/boot にマウントする.
%
%            \begin{verbatim}
%           [例] 「/boot」ディレクトリが /dev/hda1 の場合
%
%           \item mount -t ext3 /dev/hda1 /mnt/root/boot
%            \end{verbatim}
%
%
%      \end{enumerate}
\end{enumerate}

\appendix

\section{パーティションの作成とファイルシステムのフォーマット \label{sec:mkprttn-frmt}}

\subsection{パーティションを切る}

パーティションを切るには, fdisk, sfdisk, cfdiskというプログラム
を使う(fdiskにはバグが多いので, 気をつける以上に気をつけて). 例
として, fdiskでハードディスク /dev/hda に swap領域を作成する作業
を以下に示す. 

\begin{enumerate}
\item fdiskの起動

      \begin{verbatim}
      # fdisk /dev/hda
      \end{verbatim}

\item Linuxパーティションの作成

      \begin{verbatim}
      Command (m for help): n (パーティションの作成)
      Command action
        e   extended
        p   primary partition (1-4)
      p (プライマリパーティションの作成)
      Partition number (1-4): 1 (パーティション番号1の領域を作成)
      First cylinder (1-978, default 1): [Enter] (領域の最初のシリンダー)
      Using default value 1
      Last cylinder or +size or +sizeM or +sizeK (1-978, default 978): 489
        (978シリンダの半分約32MBを領域1に割り当てる)
      \end{verbatim}


\item Linux swap用パーティションの作成

      \begin{verbatim}
      Command (m for help): n (パーティションの作成)
      Command action
         e   extended
         p   primary partition (1-4)
      p (プライマリパーティションの作成)
      Partition number (1-4): 2 (パーティション番号2の領域を作成)
      First cylinder (490-978, default 490): [Enter] (領域の最初のシリンダー)
      Using default value 490
      Last cylinder or +size or +sizeM or +sizeK (490-978, default 978): [Enter]
        (残り半分約32MBを領域2に割り当てる)
      \end{verbatim}

\item システムIDをLinux swapに変更する

      \begin{verbatim}
      Command (m for help): t (パーティションシステムIDの変更)
      Partition number (1-4): 2 (パーティション番号2の領域を変更)
      Hex code (type L to list codes): 82 (パーティション番号2の領域をLinux swapにする)
      Changed system type of partition 2 to 82 (Linux swap)
      \end{verbatim}

\item 作成したパーティションの確認

      \begin{verbatim}
      Command (m for help): p (パーティションテーブルの表示)

      Disk /dev/hda: 4 heads, 32 sectors, 978 cylinders
      Units = cylinders of 128 * 512 bytes

       Device Boot   Start    End   Blocks  Id  System
      /dev/hda1            1    489    31280  83  Linux
      /dev/hda2          490    978    31296  82  Linux swap
      \end{verbatim}

\item 作成したパーティション情報の保存

     Command (m for help): w (パーティションテーブルをディスクに書き込む)
     The partition table has been altered!

\item ファイルシステムの作成

      /dev/hda1にLinuxのファイルシステムを構築します。

      \begin{verbatim}
      # mke2fs /dev/hda1
      \end{verbatim}

\item swapの作成 /dev/hda2にスワップ領域を設定します。

      \begin{verbatim}
      # mkswap /dev/hda2
      \end{verbatim}

\item 終わり.
\end{enumerate}

\subsection{各パーティションをフォーマットする}

パーティションを切り終わったら, それぞれのパーティション
をフォーマットしないと実際に利用できない. swap領域は mkswap
を使い, その他の /, /boot 等は
「mkfs -t ext3 $<$パーティション領域$>$」を利用したら良いだろう.
詳しくは調べよう.

\section{ハードディスクの不良ブロックの緊急対処}

/var/log/messages などのファイルに

\begin{verbatim}
 terminator kernel: hda: dma_intr: error=0x84 { DriveStatusError BadCRC }
 …
 kernel: hda: dma_intr: error=0x40 { UncorrectableError },LBAsect=232389,
 high=0, low=232389, sector=103848 
\end{verbatim}
等のエラーメッセージがでた場合, ハードディスクに不良セクタが出来はじめて
いると考えれます. この問題を根本的に解決するにはディスクの交換が必要でし
ょう.

ディスクの交換にはそれなりの準備と手間がかかります. とりあえず, 不良ブロ
ックを書き出し, 使用しないようにすることにしましょう.

ext2 ext3 ファイルシステムを使用している場合, 何らかの起動手段(例えばブー
トCDでrescueモードで起動)でマシンを起動し, チェックをするパーティションを
マウントしていない状態で実行します. /dev/hda3 パーティションをチェックす
る場合は, 以下のように実行します.

\begin{verbatim}
# e2fsck -c /dev/hda3
\end{verbatim}
「-c」オプションを指定すると, e2fsck は badblocks プログラムを
呼び出してファイルシステムの不良ブ ロックを探し, 見つかったもの
を不良ブロック inode に加える. このオプションが 2 つ指定される
と, 不良ブロックの スキャンは非破壊的 read-write テストを用いて
行われる.

\section{データの安全な消し方}

データの安全な消し方ってどんなことだろう. みなさんは次の
消し方をどう思いますか?

\begin{verbatim}
# rm -rf /
\end{verbatim}
危険です. これは安全なデータの消し方では
ありません. 「安全」という言葉を使う意味を説明します.

先ほどの$rm -rf /$の実行で消えたものはなんでしょう?
ファイルのデータはまだハードディスク上に記憶されてい
ます. ファイルの削除コマンドを実行したとき, ほとんど
のOSではファイルへの参照情報を削除しただけです(その
ファイルの削除をした場所のディスクスペースに他のデー
タが上書きされるまで残っています).

みなさんはプライバシーを保護したいデータをハードディスク
上においています. 例えば, 金銭上のデータ, E-mail アドレ
ス, パスワード, インターネットサーフィンの履歴があります.
ある人が古いコンピュータやハードディスクを売り, それを購
入した人が金銭上のビジネスデータ等を復旧することもあるそ
うです. 電子レンジでチンッとしたり, 煮たり, 焼いたり, 吹
き飛ばしたら確実でしょうけど, 後は使いものになりません. 

データの「安全」な消し方の意味をわかられたことでしょう.
あなたの大切なデータの復旧されるのことを不可能に近くする
唯一の方法は, あるパターンでデータを上書きすることです.
さまざまなツールがありますが, 十分だと思うだけ, 以下紹
介と実践と行きましょう.

\subsection{実践} 

GNU coreutils(Fileutils)に含まれている shred というツール
は一つのファイルだけでなく, 全てのパーティション, ハードデ
ィスクでも安全に削除することができます. 基本的な動作は25回
の上書きの繰り返しです. 上書きの回数はオプションによって増
減可能です(ハードディスクの容量によりますが, 25回の上書きの
繰り返しは非常に時間がかかります:-) ).

\begin{verbatim}
# shred -v /dev/hda
\end{verbatim}
詳しくはmanで調べてください.

プログラムddでも特定のデバイスに上書きする似た働きを行えま
す.

\begin{verbatim}
# dd if=/dev/zero or /dev/urandom of=device
\end{verbatim}
知識不足と思うなら, しない方が良いかもしれません.

\section{DVD($\pm$RW, $\pm$R, RAM)にデータを焼くには}

\subsection{作業する場所}
venus で作業してください. 注意点があります.
DVD($\pm$RW, $\pm$R)を焼くときは, autofs, supermount, subfs/submount, magicdev
などの自動オートマウント/オートプレイ機構
の制御下にある場合、必ずそれを解除してください。これが有効になっていると, 失敗
するそうです. %ですので, 焼く前に

%\begin{verbatim}
%# su -
%# /etc/init.d/autofs stop
%\end{verbatim}
%として, autofs のサービスを止めてから焼いてください.

\subsection{DVD-RAMの焼き方}

\begin{enumerate}
\item DVD-RAMをDVDドライブに入れる.
\item DVD-RAMをマウントする. その方法は以下のようにする.

      \begin{verbatim}
      % cd /mnt/dvd-ram
      \end{verbatim}

\item cp, mv 等で /mnt/dvd-ram に焼きたいデータ(ファイル等)を
      移動させる.
\item 作業が終わったら, root ユーザになって(su) 次のコマンドを
      実行し, DVD-RAMをアンマウントする.

      \begin{verbatim}
      # umount /mnt/dvd-ram
      \end{verbatim}

\end{enumerate}
以上のようにマウントしてしまえば, ハードディスクのように利用
できます. そして, データの削除も$rm$で消すことができます.

\subsection{DVD$\pm$Rにデータを焼くには}

DVD$\pm$(以下DVD)を焼くには$growifofs$を使います.
新しくDVDに書き込むときには

\begin{verbatim}
# growisofs -Z /dev/dvd <ディレクトリ>
\end{verbatim}
とします. DVDと言えば, ギガレベルの大容量データを焼くことがで
きます. 小さいファイルを焼いたあと, 残った領域がありますね?そ
の領域にデータを焼きたいときは以下のようにします.

\begin{verbatim}
# growisofs -M /dev/dvd <ディレクトリ>
\end{verbatim}
とっても便利です.

cdrecordを使ってCDを焼いた経験のある人は iso イメージを作成す
る必要さを知っているでしょうが, growisofs はその下準備は必要
ありません(isoイメージも焼ける).

\subsection{DVD$\pm$RWを焼く(未検証)}

\begin{itemize}
\item format してファイルを焼く(上書き)

      \begin{verbatim}
      $ growisofs -Z /dev/dvd -R -J /some/files
      \end{verbatim}

\item 既にファイルがあれば消さずにファイルを追加する(追記)

      \begin{verbatim}
      $ growisofs -M /dev/dvd -R -J /more/files
      \end{verbatim}

\end{itemize}

\subsection{指定したディレクトリを含めて書き込む}

growisofsの引数としてディレクトリを指定した場合, そのデ
ィレクトリ自身は含まれずに直下の内容が再帰的に書き込まれ
ます. そのディレクトリの含めて書き込みたい場合は次のよう
にします.

例えば /home と /var/lib のバックアップを取る場合で, DVD
に /home と /var/lib というディレクトリを含めて書き込みた
い場合は次のようにします(最初の書込み).

\begin{verbatim}
# growisofs -Z /dev/dvd -J -R -graft-points home=/home var/lib=/var/lib
\end{verbatim}
実際に書込むファイルやディレクトリ(右側)を「=」記号の左側で指定
したディレクトリ下に置くように指定できます. このディレクトリは実
際と異なっていても構いませんし, 存在しないディレクトリを指定する
こともできます. 例えば /home と /var/lib をバックアップを取った日
付のディレクトリ「20040528」の下に書く場合は次のようにします.

\begin{verbatim}
# growisofs -Z /dev/dvd -J -R -graft-points 20040528/home=/home 20040528/var/lib=/var/lib
\end{verbatim}
なお「=」の右側にファイルを指定し左側にディレクトリを指定する場合は, ディレク
トリ名の最後に「/」をつけるようにしてください. そうでないと右側で指定したファ
イルが左側の名前のファイル名として書き込まれます.

\subsection{ISOイメージを焼くには}
mkisofs などで既に iso-image を作ってある場合は

\begin{verbatim}
$ growisofs -Z /dev/dvd=image.iso
\end{verbatim}
とすれば焼いてくれる.

%\section{パーティションのエントリをバックアップする(ジャンク内容)}
%
%実際に行ってみたところ, 無理でした....
%
%全てのパーティションエントリー(基本領域と拡張領域にある論理
%領域の両方)を保存しよう. ここでは, hda(最初のIDEハードディス
%ク)をバックアップするとする.
%
%まず最初に, DD(GNU convert and copy)でMBRを保存しよう.
%
%\begin{verbatim}
%# cd /root
%# mkdir partition-backup
%# cd partition-backup
%# dd if=/dev/hda of=backup-hda.mbr count=1 bs=512
%\end{verbatim}
%これで512 bytesのとても大切なデータが作成される.
%さて, 拡張領域のエントリーを保存しよう.
%
%\begin{verbatim}
%# sfdisk -d /dev/hda > backup-hda.sf
%\end{verbatim}
%sfdisk は util-linux パッケージで提供されるツールです.\\
%
%[重要!]
%これらのファイルをどこか安全なところにおいておこう(例えばフロッピー, CD).
%ハードディスクの中においておかないように. もしそれらのファイルを保存してい
%るハードディスクに問題が生じると, せっかく作ったファイルにアクセスできなく
%なるかもしれない. 
%
%\subsection{バックアップからパーティションのエントリをリストアする}
%
%リストアは危険な作業なので, 気をつけましょう(データを破壊してしまう!).
%まず最初にMBR(Master Boot Record)をリストアしましょう.
%
%\begin{verbatim}
%# dd if=backup-hda.mbr of=/dev/hda
%\end{verbatim}
%そして, 拡張領域のエントリをリストアするには
%
%\begin{verbatim}
%# sfdisk /dev/hda < backup-hda.sf
%\end{verbatim}
%これで終わりです. コンピュータをリブートしましょう.
%
%\section{古いバックアップの仕方}
%\subsection{venus の /home/venus をバックアップする \label{appsub:vns-b}}
%
%作業手順を下に示します(シェルスクリプトを覗いてどんな
%動作をしているか確認してください).
%
%\begin{enumerate}
%\item 十分に容量の残っているDVD-RAMを venus のDVDドライブに
%      入れる
%\item venus に入る.
%\item cd /home/venus/adm/bin
%\item ./bkup\_venus\_DVD\_bz2
%\item あとは待つだけ.
%\end{enumerate}
%
%\subsection{venus の /home/zeus をバックアップする \label{appsub:zs-b}}
%
%作業手順を下に示します(シェルスクリプトを覗いてどんな
%動作をしているか確認してください).
%
%\begin{enumerate}
%\item 十分に容量の残っているDVD-RAMを venus のDVDドライブに
%      入れる
%\item venus に入る.
%\item cd /home/venus/adm/bin
%\item ./bkup\_zeus\_DVD\_bz2
%\item あとは待つだけ.
%\end{enumerate}
%
%\subsection{/home/venus, /home/zeus の最新のデータをバックアップしたいとき}
%
%\begin{enumerate}
%\item venus に入る.
%\item cd /home/venus/adm/bin
%\item ./bkupbcl を実行する.
%\item これで第\ref{sub:vns-b}, \ref{sub:zs-b}節の手順を行う
%      ことで, 最新のデータのバックアップができる.
%\end{enumerate}
%
\end{document}

