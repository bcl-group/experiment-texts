\documentclass{jarticle}
\usepackage{amsmath}
\usepackage{geometry}
\geometry{body={165mm,230mm}}

\def\version{2.05}
\begin{document}

\begin{center}
  {\LARGE 楽しい運動計測実習その2}
\end{center}

\section{高速度カメラによる運動計測その2}

以下の手順にしたがって、歩行運動のデータをとり、解析しなさい。
少なくとも3人のデータをとり、比較検討をおこなうこと。

\begin{center}
  注意: 取得データはデータ格納専用の外部HDDに専用フォルダを作って格納す
  ること。
\end{center}

\subsection{準備}
\subsubsection{マーカとりつけ}
頭頂・首・腰(背面×1,側面×2)・膝・かかと・足先にマーカをとりつけ、各マーカ間の距離と、脚のど
こにとりつけたがわかるように各部位の距離を測りなさい。
あとで同じ場所にマーカをとりつけられるだけの必要十分な距離データを
測定すること。

\subsubsection{関節の可動範囲}
各関節の可動域を調べなさい。
特に、踵関節については、可動範囲には能動的に動くことのできる範囲
と受動的に動くことができる範囲があることに注意して両方調べなさい。

\subsubsection{筋電センサとりつけ}
以下の各筋肉に筋電センサを取付けなさい。
取り付け場所は「表面筋電図マニュアル」を参照のこと。
\begin{enumerate}
\item 中殿筋
\item 大腿直筋
\item 外側広筋
\item 前頸骨筋
\item 半腱様筋
\item 大腿二頭筋
\item 内側腓腹筋
\item 外側腓腹筋
\item ヒラメ筋
\end{enumerate}

\subsection{運動計測}

時速3km, 時速4.5km, 時速6kmの各速度における歩行について、
腰・膝・かかと・足先の運動と筋電を計測しなさい。% 1人あたり2セット行い,1,2セット間で仙骨体操もしくは屈伸体操を行うこと。
% 計測には\verb|gettv体操.sh|コマンドを使う。
% 詳細はマニュアル「ポジションセンサを用いた運動計測の方法」参照。
\begin{itemize}
\item 「走る」のではなく、「歩く」こと。
\item 各速度での歩行に慣れるための練習時間を必ず設けること。
\item 歩き始めの10歩程度は非定常状態になることが多いので,データ処理に
  は使わないこと。
\item 少なくとも数十歩のデータに対する統計処理を行えるようにデータは取
  得しておくこと。
\item 遊脚の開始と終了がわかるようにカメラの配置を工夫すること。
\item 取得した各関節の位置データは,各人のホームディレクトリに転送して
  以下のように解析を行いなさい。
\end{itemize}

\subsection{運動軌道計測結果の解析}

% \verb|gettva.sh|コマンドを使うと様々なデータ処理が行われるが、
% \verb|trajectory/|ディレクトリの下にできる各マーカの位置データ
% \verb|*_tgt?-m-smth.dat|を用いて以下を行う(ここで、\verb|*|はデータ
% 取得時に指定した名前、\verb|?|はマーカ番号)。

\subsubsection{前処理}

計測によって得られた軌道データを、すでにつくってあるシェルスクリプ
  ト\verb|getdat.sh|によって,ヘッダ部とデータ部に分けなさい。
ここで得られたヘッダファイル名を以下では仮りに\verb|info-joints.dat|,
データファイル名を\verb|pos-joints.dat|とする。

\subsubsection{ローパスフィルタによる軌道処理}
計測したEMG信号は前回の実験と同様に適当な大きさのウィンドウ幅で
平滑化する。゜

\subsubsection{ローパスフィルタによる軌道データ処理}
\begin{enumerate}
\item 取得した軌道計測データにはノイズがのっているので、ローパスフィルタを
  通して処理する。
  計測のサンプリング周波数を$f$[Hz]としたとき、
  ローバスフィルタのしゃ断周波数は最大でどのような値に設定すべきか? 
  サンプリング定理に従って答えなさい。

  ただし、ヒトの運動計測の際のしゃ断周波数は通常7-10Hz程度に設定する。
\item 指定したデータファイルの各座標データに、ローパスフィルタをかける
  シェルスクリプト\verb|lpf.sh|を作りなさい。その仕様は以下の通りとす
  る。
  \begin{enumerate}
  \item 入力データファイル名が\verb|pos-joints.dat|ならば、
    出力データファイル名は\verb|spos-joints.dat|とし、
    そのフォーマットは入力データファイルと同じとする。
  \item しゃ断周波数は\verb|lpf.sh|内のマクロ変数で指定するようにすること。
  \item ローパスフィルタは研究室のcvsで管理されているバターワースフィ
    ルタ(レポジトリ名はfilters,cvsの使いかたは「UNIX/LINUXの基本操作」
    を参照すること)を\verb|lpf.sh|から呼び出すようにしなさい。
    用意されているプログラムは1列のみのデータを処理するものであることに気
    をつけて利用すること。具体的な手順は以下の通り。
    \begin{enumerate}
    \item 入力データファイル(例えば\verb|pos-joints.dat|)の各列をcutを
      用いて、それぞれを例えば
      \begin{quote}
        \verb|tmp-time-joints.dat|, 
        \verb|tmp-1-pos-joints.dat|, 
        \verb|tmp-2-pos-joints.dat|
      \end{quote}
      といった名前の別々の一時ファイルにする。

      ちなみに、シェルスクリプト中での算術演算は以下のようにする。
\begin{verbatim}
NUM=1                   #変数NUMに1が代入される
NUM =`expr ${NUM} + 1`  #変数NUMが2になる
TMPNAME=tmp-${NUM}      #変数TMPNAMEを"tmp-2"にした
\end{verbatim}
    \item 時刻データをのぞく各データそれぞれをバターワースフィルタで処
      理し,\verb|tmp-1-spos-joints.dat|といった名前にする。
    \item 時刻データとバターワース処理後の各データファイルを再結合して、
      \verb|spos-joints.dat|をつくる。
      \verb|spos-joints.dat|のフォーマットは入力ファイルであった
      \verb|pos-joints.dat|と同じである。
    \item 一時データ\verb|tmp-*|を削除する。
    \end{enumerate}
  \item バターワースフィルタを呼び出すときに必要となる
    サンプリング周波数は、先につくってある \verb|info-joints.dat| から
    抽出する。
  \item 入力データファイルに含まれている計測点数がデータ処理に
    必要であれば、\verb|info-joints.dat| から抽出して利用しなさい。
  \end{enumerate}
\end{enumerate}

\subsubsection{相対位置の計算}
\begin{enumerate}
\item 以下のような仕様のプログラム\verb|pos2rel.c|を作りなさい。
  \begin{enumerate}
  \item ローパスフィルタを通して得られた位置データファイル\verb|spos-joints.dat|
    を読み込んで,各時刻の腰関節位置に対する他の各マーカ位置の相対位置
    を計算して出力する。
    ある時刻の位置データA,Bの値をそれぞれ$(x_A,y_A)$, $(x_B,y_B)$とす
    るとAに対するBの相対位置は$(x_B-x_A,y_B-y_A)$で与えられる。
  \item 
    出力ファイルのフォーマットは、入力ファイルに準ずる。
    ただし、相対座標にするため、腰の位置データはなくなる。
    すなわち、腰関節位置を($x_1,y_1$)とするならば、出力データのフォーマットは以下の通り。
\begin{quote}
時刻、$x_2-x_1, x_3-x_1,..., y_2-y_1, y_3-y_1$,...
\end{quote}
  \item 入力ファイルが\verb|spos-joints.dat|ならば、出力ファイル名は
    \verb|rel-spos-joints.dat|とする。
%   \item 各ファイル名およびサンプリング周波数を、以下のように実行時
%   のオプションとして与える。
% \begin{verbatim}
% $ pos2rel ファイルA ファイルB 出力ファイル名 サンプリング周波数
% \end{verbatim}
\end{enumerate}
% \item コマンド\verb|pos2rel|と\verb|pos2vel|を呼び出して、腰関節に対す
%   る各マーカ位置の相対座標と相対速度を出力するシェルスクリプト
%   \verb|getrel.sh|を作りなさい。 
%   \begin{enumerate}
%   \item \verb|getrel.sh|では、for文を用いて,各位置データ
%     (\verb|*_tgt?-m-smth.dat|)を順に呼び出し、相対位置データ
%     (\verb|rel-*_tgt?-m.dat|)と相対速度データ
%     (\verb|rel-*_tgt?-vel.dat|)を出力するものとする。
%   \item サンプリング周波数はデータ取得時に作られているディレクトリ
%   \verb|original/|以下にあるファイル\verb|*_info|を利用して抽出するこ
%   と。 
%   \end{enumerate}
\end{enumerate}


\subsubsection{速度の計算}
\begin{enumerate}
\item ローパスによる処理後、相対位置データを保存したデータファイルを、
すでに作成している\verb|extract.sh|を改良して、各計測点毎の相対位置デー
タを抽出できるようにしなさい。
入力ファイルが\verb|rel-spos-joints.dat|ならば、出力ファイル名は計測点
IDを使って\verb|1-rel-spos-joints.dat|, \verb|2-rel-spos-joints.dat|,...とする。
\item 各計測点毎の位置データファイルを読み込んで、速度データを出力する
プログラム\verb|pos2vel.c|を作りなさい。以下はその仕様である。
  \begin{enumerate}
  \item 「速さ」とは速度の大きさである。
  \item 実行時の引数は以下の通り。
\begin{verbatim}
    $ pos2vel 入力ファイル名 出力ファイル名 サンプリング周波数
\end{verbatim}
  \item 速度データを求めるための差分計算には、引数としてうけとったサン
    プリング周波数を用いる。
    \textbf{入力データの時刻フィールドの値は丸め誤差のため、その時間間
      隔は一定になってないため、時刻データを速度計算に使ってはいけない}
  \item 出力ファイルのフォーマットは以下の通りである。
\begin{verbatim}
-----------------------------------------
  時刻  速度のx成分  速度のy成分  速さ
-----------------------------------------
\end{verbatim}
  \end{enumerate}
\item コマンド\verb|pos2vel|を呼び出して、各マーカの位置データをもとに、
  各速度データを計算するシェルスクリプト\verb|getvel.sh|を作り
  なさい。
  \begin{enumerate}
  \item for文を用いて,各位置データ(\verb|?-rel-pos-joints.dat|)を順に
    呼び出し、速度データ(\verb|?-re-lvel-joints.dat|)を出力する。
  \item サンプリング周波数は先につくってある
    \verb|info-joints.dat|からサンプリング周波数から抽出する。
  \end{enumerate}
\end{enumerate}

% \subsubsection{相対位置・速度の計算}
% \begin{enumerate}
% \item 以下のような仕様のプログラム\verb|pos2rel.c|を作りなさい。
%   \begin{enumerate}
%   \item ローパスフィルタを通して得られたマーカの位置データファイルA,B
%     を読み込んで,各時刻のAに対するBの相対位置を出力する。(ある時刻の
%     位置データA,Bの値をそれぞれ$(x_A,y_A)$, $(x_B,y_B)$とするとAに対するB
%     の相対位置は$(x_B-x_A,y_B-y_A)$で与えられる。
%   \item 各ファイル名およびサンプリング周波数を、以下のように実行時
%   のオプションとして与える。
% \begin{verbatim}
% $ pos2rel ファイルA ファイルB 出力ファイル名 サンプリング周波数
% \end{verbatim}
% \item 出力ファイルのフォーマット
% \begin{verbatim}
% -----------------------------------------
%   時刻 相対位置のx軸成分 相対位置のy軸成分
% -----------------------------------------
% \end{verbatim}
%   \end{enumerate}
% \item コマンド\verb|pos2rel|と\verb|pos2vel|を呼び出して、腰関節に対す
%   る各マーカ位置の相対座標と相対速度を出力するシェルスクリプト
%   \verb|getrel.sh|を作りなさい。 
%   \begin{enumerate}
%   \item \verb|getrel.sh|では、for文を用いて,各位置データ
%     (\verb|*_tgt?-m-smth.dat|)を順に呼び出し、相対位置データ
%     (\verb|rel-*_tgt?-m.dat|)と相対速度データ
%     (\verb|rel-*_tgt?-vel.dat|)を出力するものとする。
%   \item サンプリング周波数はデータ取得時に作られているディレクトリ
%   \verb|original/|以下にあるファイル\verb|*_info|を利用して抽出するこ
%   と。 
%   \end{enumerate}
% \end{enumerate}

\subsubsection{運動軌道の表示}

各関節の絶対座標および相対座標に関して以下のグラフを作成し、その特徴
を考察しなさい。
脚軌道の特徴や,速度に応じたその変化がどのような筋活動の結果によるのか
も考察しなさい。
\begin{enumerate}
\item 水平-垂直位置
\item 時間-水平位置
\item 時間-垂直位置
\item 時間-速度
\item 時間-速さ
\item 歩行速度-足の運動周期
\item 歩行速度-足のふり幅
\item その他?
\end{enumerate}

以下は解析や考察における着目点の例。
\begin{itemize}
\item どこが接地相でどこが遊脚相? (軌道からどう判断する?)
\item 筋活動はどのような瞬間に見られる?
\item 個人差はどこにある?
\item 被験者によらない共通の特徴はどこにある?
\item 速度による運動軌道の変化はどのような点にある?
\item 歩行と走行に運動軌道の違いはあるか?
\item かかと関節角度は、常に能動的な可動域範囲にあるか?
\end{itemize}

%% \subsubsection{歩行パラメータ}
%% 以下の歩行パラメータが各歩容および速度に対してどのように変わっているか
%% を表もしくはグラフにまとめる。(どちらが見やすいか考える)。
%% 各パラメータについて、数周期のデータに対する平均および分散を求めること。
%% \begin{enumerate}
%% \item 歩行周期
%% \item 遊脚時間
%% \item ストライド長(一歩行周期の間にすすむ距離)
%% \item ステップ長(接地相の間に胴体が進む距離)
%% \end{enumerate}

%% \newpage
%% \section{第5日:いろいろな運動計測}
%% なにか、運動計測テーマを決めて,自由に解析しましょう。
%% 計測は,ビデオを用いても、ポジションセンサを用いてもどちらでもOKです。

\subsection{藤井からのお願い}
以下の内容について調べて下さい.質問いつでも受け付けます!
\begin{enumerate}
\item 矢状面において頭頂と足の軌道を図示して考察してください.
\item 前額面において腰(背面)に対する首の相対位置を求め,水平方向の分散から上半身のぶれ具合を見てみましょう.
\item 外転筋の働くタイミング(接地時?遊脚時?)と活動度を調べて下さい.
% \item 以上の計測結果が体操によって変化するのか.また変化があればそれの意味することを考えてみてください。
\end{enumerate}

\section{プレゼンテーション}
LibreOffice のプレゼンテーションソフトを使って、プレゼンテーションの資
料を作りましょう。
\begin{enumerate}
\item 計測実習の内容(何を目的として、どのような実験をして、どんな結果
  が得られ,どんな考察をしたか)を16分程度(一人4分程度)で発表できるよう
  にすること。
\item 目的を自由に設定して、それに対応したストーリーになるように手法・
  結果を話すこと。
\item 筋活動に関する結果・考察に触れること
\item 15分の発表ならば,スライドのページは15枚程度。
\item 作成したグラフは厳選して必要最小限のものを見せること。
\item 聞き手が楽しく聞けるように、よく内容を吟味すること。
\item 発表前に,必ず話す練習をすること。
\end{enumerate}

\end{document}
