\documentclass{jarticle}
\usepackage{amsmath}
\usepackage{geometry}
\geometry{body={165mm,230mm}}

\def\version{3.01}
\begin{document}

\begin{center}
  {\LARGE 楽しい運動計測実習: 歩行計測編}
\end{center}
\begin{flushright}
  \today
  \end{flushright}

\section{高速度カメラによる歩行計測}

以下の手順にしたがって、歩行運動のデータをとり、解析しなさい。
少なくとも3人のデータをとり、比較検討をおこなうこと。

% \begin{center}
%   注意: 取得データはデータ格納専用の外部HDDに専用フォルダを作って格納すること。
% \end{center}

\subsection{準備}

以下のように反射マーカと筋電センサを被験者にとりつける。
ただし,取付箇所は以下に記載されている指示から変更する場合もある。

\subsubsection{反射マーカの貼付}
股関節・膝関節・足関節・足先(第五指中足骨基部)にマーカを貼付しなさい。
また,各マーカ間の距離を測っておき,後のデータ解析の際に変な値になっていないか確認すること。

% \subsubsection{関節の可動範囲}
% 各関節の可動域を調べなさい。
% 特に、踵関節については、可動範囲には能動的に動くことのできる範囲と受動的に動くことができる範囲があることに注意して両方調べなさい。

\subsubsection{筋電センサとりつけ}
以下の各筋肉に筋電センサを取付けなさい。
取り付け場所は「表面筋電図マニュアル」を参照のこと。
\begin{enumerate}
\item 中殿筋
\item 大腿直筋
\item 外側広筋
\item 前頸骨筋
\item 半腱様筋
\item 大腿二頭筋
\item 内側腓腹筋
\item 外側腓腹筋
\item ヒラメ筋
\end{enumerate}

\subsection{運動計測}

時速3km, 時速4.5km, 時速6kmの各速度における歩行について、
股関節・膝関節・足関節・足先の運動と筋電を計測しなさい。
% 1人あたり2セット行い,1,2セット間で仙骨体操もしくは屈伸体操を行うこと。
% 計測には\verb|gettv体操.sh|コマンドを使う。
% 詳細はマニュアル「ポジションセンサを用いた運動計測の方法」参照。
\begin{itemize}
\item 「走る」のではなく、「歩く」こと。
\item 各速度での歩行に慣れるための練習時間を必ず設けること。
\item 歩き始めの10歩程度は非定常状態になることが多いので,データ処理には使わないこと。
\item 少なくとも数十歩のデータに対する統計処理を行えるようにデータは取得しておくこと。
\end{itemize}

\section{データ処理}
取得データは,各人のホームディレクトリに転送して以下のように解析を行いなさい。

\subsection{運動軌道のフィルタ処理}
モーションキャプチャーシステムにより取得した運動軌道に対して以下の処理を行う。
\begin{enumerate}
  \item 低周波フィルタによるノイズ処理
  \item 数値差分による速度の計算
\end{enumerate}



\subsection{筋電信号のフィルタ処理}

以下のように処理する。
\begin{enumerate}
  \item 高周波フィルタ
  \item 整流
  \item 低周波フィルタ(平滑化を行う場合もあるが本実験では低周波フィルタを使う)
\end{enumerate}

\subsection{グラフ化}

各関節の絶対座標および相対座標(股関節に対する膝関節,足関節,足先の位置)に関して以下のグラフを作成し、その特徴を考察しなさい。
脚軌道の特徴や,速度に応じたその変化がどのような筋活動の結果によるのかも考察しなさい。
\begin{enumerate}
\item 水平-垂直位置
\item 時間-水平位置
\item 時間-垂直位置
\item 時間-速度
\item 股関節,膝関節,足関節の各角度の三次元空間内での変化
\item 歩行速度-足の運動周期
\item 歩行速度-足のふり幅
\item 時間-筋活動
\item その他?
\end{enumerate}

\subsection{運動軌道および筋電位の解析}

\begin{enumerate}
  \item 関節軌道(股関節に対する相対軌道)に対して主成分分析(Principal Component Analysis, PCA)を行いなさい。
  \begin{enumerate}
    \item いくつの主成分で寄与率は90\%を超えるだろうか? 
    \item 脚運動軌道はどのような基本的な運動に構成されているだろう?
    \item 各主成分の寄与は,歩行の各タイミングに応じてどのように変化するだろうか。
  \end{enumerate}
  
  \item 筋電位に対して非負値行列因子分析(Nonnegative Matrix Factorization, NMF)を行いなさい。
  \begin{enumerate}
    \item いくつの主成分で寄与率(Variance Accounted For)は90\%を超えるだろうか? 
    \item 取得した筋活動は,主に何種類の筋シナジーで実現されているだろうか?
    \item 主要な筋シナジーの役割はどのようなものだろうか?
    \item 各筋シナジーの活動度は,歩行の各タイミングに応じてどのように変化するだろうか。
  \end{enumerate}
\end{enumerate}

\subsection{考察}
各自の発見をまとめなさい。ここまでに作成した全てのデータ・グラフを最終的な報告に使う必要はないが,
PCA, NMFによる解析結果は必ず含めること。また,筋電位だけとか,運動軌道だけの議論にならないこと。
以下は解析や考察における着目点の例。

\begin{itemize}
\item どこが接地相でどこが遊脚相? (軌道からどう判断する?)
\item 各筋活動(もしくは筋シナジーの活動)はどのような瞬間に見られる?
\item 被験者によらない共通の特徴はどこにある?
\item 個人差はどこにある?
\item 速度による運動軌道の変化はどのような点にある?
\item 脚運動周期やステップ長(接地相の間に胴体が進む距離)と歩行速度の関係は?
\end{itemize}

%% \subsubsection{歩行パラメータ}
%% 以下の歩行パラメータが各歩容および速度に対してどのように変わっているか
%% を表もしくはグラフにまとめる。(どちらが見やすいか考える)。
%% 各パラメータについて、数周期のデータに対する平均および分散を求めること。
%% \begin{enumerate}
%% \item 歩行周期
%% \item 遊脚時間
%% \item ストライド長(一歩行周期の間にすすむ距離)
%% \item ステップ長(接地相の間に胴体が進む距離)
%% \end{enumerate}

%% \newpage
%% \section{第5日:いろいろな運動計測}
%% なにか、運動計測テーマを決めて,自由に解析しましょう。
%% 計測は,ビデオを用いても、ポジションセンサを用いてもどちらでもOKです。

\subsection{藤井からのお願い}
以下の内容について調べて下さい.質問いつでも受け付けます!
\begin{enumerate}
\item 矢状面において頭頂と足の軌道を図示して考察してください.
\item 前額面において腰(背面)に対する首の相対位置を求め,水平方向の分散から上半身のぶれ具合を見てみましょう.
\item 外転筋の働くタイミング(接地時?遊脚時?)と活動度を調べて下さい.
% \item 以上の計測結果が体操によって変化するのか.また変化があればそれの意味することを考えてみてください。
\end{enumerate}

\section{プレゼンテーション}
以下のように成果発表を行いましょう。

\begin{enumerate}
\item 計測実習の内容(何を目的として、どのような実験をして、どんな結果
  が得られ,どんな考察をしたか)を15分程度で発表できるようにまとめること。
\item 目的を自由に設定して、それに対応したストーリーになるように手法・結果を話すこと。
\item 15分の発表ならば,スライドのページは15枚程度。
\item 作成したグラフは厳選して必要最小限のものを見せること。
\item 聞き手が楽しく聞けるように、よく内容を吟味すること。
\item 発表前に,必ず話す練習をすること。
\end{enumerate}

\end{document}
