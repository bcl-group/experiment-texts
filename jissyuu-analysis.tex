\documentclass{jsarticle}
\usepackage{amsmath}
\usepackage{geometry}
\geometry{body={165mm,230mm}}

\def\version{3.2}
\begin{document}

\begin{center}
  {\LARGE 楽しい運動計測実習 データ解析編(ver.\version)}
\end{center}
\begin{flushright}
\today
\end{flushright}

以下のデータ解析用プログラムを作りなさい。
プログラミング言語は好きなものを用いてよいが,シェルスクリプトを作成するように指示があるときにはその通り作成すること。

%\section{マニュアル作成(グループ作業): 目標2日間}
% GitHubのbcl-groupグループ(https://github.com/bcl-group)にリポジトリ"exp-manuals"を作成して,機器操作マニュアルを共同編集で作りましょう。
% \begin{enumerate}
%   \item bcl-groupへの参加招待メールが届いてるはずなので,確認・承認して下さい。
%   \item リポジトリを作ったら,西井に連絡して下さい
%   \item マニュアルは筋電計用(emg-manual.md)とモーションキャプチャ(mc-manual.md)の二種類を作成
%   \item 画像ファイルは各マニュアル毎に異なるディレクトリ(images-emg, images-mc)に格納
%   \item 書式は GitHub Flavoured Markdown で記述(googleで調べること)
%   \item 文章はなるべく\textbf{簡潔に,しかし具体的に}書くこと。
%   \item 作ったマニュアルは,GitHub上で(も)見ることが出来る。
% \end{enumerate}

\section{バンドパスフィルタについて}  
一般に計測信号にはノイズがのるので,S/N比をよくするためにバンドパスフィルタを用いる。
ただし,単純にバンドパスフィルターをかけて得られるデータは元データに対して少し時間ズレを起こすことが多い。
そこで,順方向と逆方向の両方から一回づつフィルター処理することで時間的なズレを補正する。
この処理は,pythonならばscipyのfiltfilt関数を使えば自動で行われる。

\section{筋電データの処理: 目標3日間}

\subsection{周波数分布の確認}

\begin{enumerate}
\item 筋電位の周波数分布を調べなさい
\item ノイズと思われる特徴的な周波数はないか確認しない
\item 筋電位の周波数分布は負荷に応じて変化するか確認しなさい。
\end{enumerate}

\subsection{フィルタ処理}

\begin{enumerate}
\item \textbf{高周波フィルタ}: 筋電位に遮断周波数10 Hzの高周波フィルタ(3次バターワースフィルタ)をかける。(バンドパスフィルタを用いて高周波ノイズを遮断する場合もある。)
\item \textbf{整流}; 筋電位$E(t)$を整流する(絶対値をとる)。
\item \textbf{低周波フィルタ or 平滑化} (どちらを用いるかは指示に従うこと)
  \begin{enumerate}
    \item 平滑化の場合は,適当な幅$\Delta T$の窓内での積分値を求め, 筋肉の活動度を見る指標とする。
    つまり, 時刻$t$における筋肉の活動度($a(t)$)は以下で評価することになる。
    \begin{align}
      a(t)=\frac{1}{\Delta T}\int_{t-\Delta T/2}^{t+\Delta T/2}|E(t)|dt\notag
    \end{align}
    PythonやRの場合,「移動平均」でgoogle検索すれば計算に利用できるライブラリがすぐに見つかる。ただし,移動平均をとることで時間情報がずれることがあるので,注意すること。
    \item 低周波フィルタを用いる場合は,上記の「高周波フィルタ」の項目を参考にし,遮断周波数は 15 Hzに設定しなさい。 
    ただし,必要に応じてこの遮断周波数は変更しても良い。
  \end{enumerate}
\item \textbf{規格化}: 本実験では行わないが,一般には必要になることが多い。
\end{enumerate}

\textbf{各処理過程におけるデータはそれぞれグラフ表示により確認し,処理が適切に行われているかを確認}すること。

\subsection{運動軌道データの処理: 目標3日間}
\subsubsection{運動軌道データの切取り}

\begin{enumerate}
\item 計測によって得られた身体軌道のデータの内容を確認し, フォーマットとデータ
  (数値)の意味を把握しなさい。
  フォーマットは,使うソフトウェアの種類によって異なるが,大抵以下のようになっている。
  \begin{enumerate}
  \item ヘッダ部(始め):計測条件, データの各列のタイトル
  \item データ部: シーン番号、マーカID、x座標, y座標, z座標等
  \item テール部(最後): 計測データを統計処理したデータが格納されている。
  \end{enumerate}
\item データファイルの名前を\verb|joints.csv|とするとき、このファイル
  からデータ部のみを取り出した\verb|pos-joints.dat|と、
  それ以外(ヘッダ部とテール部)のみを取り出したファイル\verb|info-joints.dat|
  を作成するシェルスクリプト\verb|getdat.sh|を作りなさい。
  各出力ファイルの仕様は以下の通りとする。
  \begin{enumerate}
  \item \verb|pos-joints.dat|は、データ部をそのままとり出す。
% \begin{verbatim}
% シーン,計測点1x,計測点2x,...,計測点1y,計測点2y,...
% \end{verbatim}
  \item \verb|info-joints.dat|は、データ部以外を取り出す。
    (もしあれば)ダブルクォーテーション(")は取り除く。
%    各行はじめの2フィールドのみを取り出す。(sedおよびcutコマンドを利用する)
  \end{enumerate}
  取得データファイルにおいて、データ部の行頭が
  数字であることを利用すれば,egrepコマンドを使ってデータ部とそれ以外をそ
  れぞれ抽出できる。
\end{enumerate}


\subsubsection{運動軌道データの処理}

モーションキャプチャーシステムにより取得した運動軌道には一般にノイズがのっているため,低周波フィルタによるノイズ処理を行う。
低周波フィルタには三次バターワースフィルタを用いることが一般的である。
遮断周波数は8-10 Hz程度にすることが多いが,ノイズの強さや,注目する運動の速さに応じて変更する。

運動軌道データに高周波フィルタ(3次バターワースフィルタ)をかけるプログラムを作りなさい。
遮断周波数が5 Hz, 10 Hzの場合について,ノイズが適切に除去されることを確認しなさい。



% \subsubsection{運動軌道データの処理}

% データファイル\verb|pos-joints.dat|から、各関節
% の各座標データを抽出する処理を自動化したい。
% \begin{enumerate}
% \item 以下のようなプログラム\verb|extract.c|(言語に応じて\verb|extract.py|,\verb|extract.R|等)を作りなさい。
%   \begin{enumerate}
%   \item 実行時には以下のような引数を指定できるようにする。
% \begin{verbatim}
%     $ extract <サンプリング周波数> <マーカ数> <入力データファイル名>
% \end{verbatim}
%   \item 出力データに書き込む「時刻」は、引数で与えた「サンプリング周波数」
%     と、入力データの「シーン」番号により計算する。
%   \item 入力ファイルが\verb|name.dat|のとき、出力ファイル名は計測点
%     IDを使って\verb|1-name.dat|, \verb|2-name.dat|,...とする。
%     もし、入力ファイル名が\verb|pos-joints.dat|ならば、出力ファイル名は計測点
%     IDを使って\verb|1-pos-joints.dat|,
%     \verb|2-pos-joints.dat|,...である。
%   \item 各出力データファイルのフォーマットは以下の通り。
% \begin{verbatim}
% 時刻, x座標, y座標, z座標
% \end{verbatim}
%   \end{enumerate}
% \item  以下のようなシェルスクリプト\verb|extract.sh|を作りなさい。
%   その仕様は以下の通りとする。
%   \begin{enumerate}
%   \item 実行時には以下のような引数を指定できるようにする。
% \begin{verbatim}
% $ extract.sh <入力データファイル名>
% \end{verbatim}
%   \item 前問の\verb|extract|コマンドを、サンプリング周波数、計測
%     点数、入力データファイル名を指定して呼び出して、各関節のxyzデータを
%     抽出する。
%   \item サンプリング周波数とマーカ数は先につくってある
%     \verb|info-*.dat|から抽出する。
%     (例えばgrepとcutを用いて抽出できる。
%     シェルスクリプトで,あるコマンド\verb|cmd|の実行結果を変数
%     \verb|CMD|に格納するには\verb|CMD=`cmd`とすればOK|)

%   \end{enumerate}
% \item 以下のようなシェルスクリプト\verb|cut23|をつくりなさい。
% \begin{enumerate}
% \item 以下のように実行したら,入力ファイルの第2,3,4フィールド(xyz座標デー
%   タ)のみを抽出したファイルを出力する。(cutコマンドを使う)
% \begin{verbatim}
% $ cut23 1-pos-joints.dat
% \end{verbatim}
% \item 上記のように入力ファイル名が\verb|1-pos-joints.dat|ならば、出力
%   ファイル名は\verb|1-xy-joints.dat|とする。
% \end{enumerate}
% \end{enumerate}

\subsection{プレゼン作成: 目標3日間}
\begin{enumerate}
\item 実験結果をプレゼンにまとめなさい。
\item 実験の目的、手法、結果、考察を自分なりにまとめること。
\item 生データや処理したデータはグラフ(x-t,y-t,x-y等)にして考察しなさい。
\item 結果をわかりやすく示す図表を作ること。
\item \textbf{グラフ作成に表計算ソフトは使わず},pythonやRのグラフ作成ライブラリを使いなさい。
\end{enumerate}

% \subsection{レポート作成: 目標2日間}
% \begin{enumerate}
% \item \LaTeX を使って、実験結果をレポートにしなさい。
% \item 実験の目的、手法、結果、考察をきちんと書くこと。  ただし,考察は数行程度でOK。
% \item 結果をわかりやすく示す図表を作ること。 絵を描くにはLibreOffice等を使う。
%   表は\LaTeX の表組機能を使うこと。
% \item 図表の引用は\verb|\label|,\verb|\ref|を使った \LaTeX の自動引用機能を使う事。
% \item 掲載した図表は必ず本文中で引用して説明すること。
% \item \LaTeX の使い方は裏ページや,本棚の書籍等を参照のこと。
% \item \textbf{グラフ作成に表計算ソフトは使わず},pythonやRのグラフ作成ライブラリを使いなさい。
% \end{enumerate}

\end{document}
