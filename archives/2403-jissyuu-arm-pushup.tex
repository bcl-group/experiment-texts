\documentclass{jsarticle}
\usepackage{amsmath}
\usepackage{geometry}
\geometry{body={165mm,230mm}}

\def\version{3.4}
\begin{document}

\begin{center}
  {\LARGE 楽しい運動計測実習: 基礎編(腕立て伏せバージョン)}
\end{center}
\begin{flushright}
\today
\end{flushright}

\section*{実習での注意}
\begin{itemize}
\item 計測実験の時には、「データをとっては確認」を繰り返すのが王道。
  計測データを全て計測してから計測時の不備に気づいて、全データの取り直しにならないように注意すること。
\item 以下の実験では, 原則として全員のデータを取って解析してください。個人差が無いかを確認することは重要です。
\item 解析プログラムは各人で作成後, お互いにその出力が同じになっているか相互に確認すること。バグ取りは重要です。
\item 取得データはデータ格納専用の外部HDD等に専用フォルダを作って格納すること。
  計測用パソコンのローカルディスクには放置しないでください。身元不明ファイルになってしまい, 後で困ります。
% \item 運動計測実験では,身長・体重等の身体パラメータがわかったほうが考
%   察に役立つこともありますが,体重等を秘密にしたい方は無理に測らなくて
%   もOKです。
\end{itemize}

\section{実験1: 負荷と筋電位}

\subsection{必要な機器等}

筋電計測用パソコン,筋電センサ一式,実験で手に持つ重り(2kg, 4kg, 6kg),被験者チェックシート

\subsection{実験}

\begin{enumerate}
  \item 身長と体重を記録しなさい。
  \item 上腕二頭筋と上腕三頭筋に筋電センサを貼付しなさい。
  \item 筋電位の周波数分布ははおおむね 5 Hz から 500 Hz である。
  したがって,計測のサンプリング周波数は 1000 Hz 以上である必要がある。今回は1000 Hzで計測する。
  \item 上腕は鉛直下向き, 前腕を前に水平に出した状態で, いろいろな重り(重りなし, 2kg, 4kg, 6kg)を 10 秒間持ったときの上肢の筋電位(EMG)を計測しなさい。
\end{enumerate}  

取得データの処理については別紙「楽しい運動計測実習,データ解析編」を見ること。

\subsection{プレゼンテーション/レポート}

負荷と,上腕二頭筋および上腕三頭筋の筋電位の大きさとの関係を解析し,その結果及び自分なりの発見をレポート(and/or プレゼンテーション)にまとめなさい。

計測データに対しては,いろいろなデータ処理や可視化の方法があり得る。データ背後にある法則性についての仮説を考え,それを検証するためにはどのようなデータ処理をしてどのようなグラフを作れば良いかよく考えること。以下は主な考察ポイント。

\begin{enumerate}
  \item 平均的な筋活動はおもりの重量とともに線形に増加するか,非線形に増加するか
  \item 重りを持って姿勢維持を続けた場合,筋活動には時間に伴う変化はあるか
  \item 以上については,平均値だけでなく標準偏差についても議論すること
  \item (optional) 負荷に伴う周波数分布の変化はあるだろうか
  \item その他発見はあるか
\end{enumerate}

プレゼンテーション等の資料を作る時には,作ったグラフをやみくもに並べるのではなく,自分の主張を根拠とともに明確に伝えるには,どのグラフ(何についての関係性)を,どのように(縦軸や横軸のレンジ等)作れば良いかを良く考えること。


\section{実験2: 腕立て伏せの動作計測}

\subsection{必要な機器等}

メトロノーム(パソコンかスマホ利用),筋電センサ一式,筋電計測用パソコン,モーションキャプチャ一式,モーションキャプチャ用パソコン,同期ユニット,被験者チェックシート

\subsection{実験}

\begin{enumerate}
\item 手首, 肘関節, 肩関節の位置に反射マーカを,上腕二頭筋, 上腕三頭筋, 大胸筋上部, 三角筋前部に筋電計をとりつけなさい。(チェックシート参照)
% 上腕二頭筋は肘を曲げる動作
% 上腕三頭筋は肘を伸ばす動作
% 大胸筋上部は内側斜め上に腕を振り上げる動作
% 三角筋前部は前方向に腕を振り上げる動作
\item モーションキャプチャのサンプリングレートは 180 fps にすること。
%\item 各関節位置の軌跡はMoveTRで抽出する。
%\item 得られたデータは自分のホームに転送して解析すること。
\item 腕立て伏せを行う際の上肢関節軌道と筋電位(EMG)を以下の各条件で計測しなさい。
ペースがわかるように,メトロノームを1秒ごとに鳴らしておきなさい。
負荷が高いと感じる被験者は膝をついても良い。
また,各条件ごとに休憩をとりなさい。
  \begin{enumerate}
    \item 両手の間隔を肩幅と同じとし,通常ペース(1 秒で下げて 1 秒で上げる)で5回
    \item 両手の間隔を肩幅より拳2つ分程度広くして,通常ペースで5回
    \item 両手の間隔を肩幅より拳2つ分程度狭くして,通常ペースで5回
    \item 両手の間隔を肩幅と同じとし,遅いペース(3秒で下げて3秒で上げる)で5回
  \end{enumerate}
\end{enumerate}

\subsection{プレゼンテーション/レポート}

以下をレポート(and/or プレゼンテーション)にまとめなさい。
\begin{enumerate}
  \item 各筋活動は,腕立て伏せの往復運動に対してどのタイミングでおきるだろうか(力学の問題として捉えて考えること。)
  \item 腕立て伏せの両手の幅を変えたとき, 活動する筋肉や活動の大きさには変化があるだろうか。その理由はなにか。
  \item 腕立て伏せのペースを変えるとき,筋活動は周期が変わるのみか。各筋活動のタイミング(位相)に変化はあるだろうか。一周期あたりの活動時間に変化はあるだろうか。その理由はなにか。
  \item その他,自分なりの考察や発見を説明しなさい。
\end{enumerate}

\end{document}
